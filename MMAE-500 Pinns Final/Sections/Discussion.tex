Overall, the results suggest that the neural network is capable of accurately predicting the behavior of the diffusion equation for a range of diffusion coefficients, which could have applications in fields such as physics, chemistry, and engineering. However, it should be noted that the quality of the results may depend on factors such as the complexity of the system being modeled, the quality of the training data, and the architecture and hyperparameters of the neural network.

The predicted temperature distribution plots shown Figures(\ref{fig:hf}, \ref{fig:tmd}, \ref{fig:t01}) that the PiNN model is able to accurately predict the temperature distribution for different values of the diffusion coefficient. The plots also indicate that the temperature distribution is symmetric along the x and y axes. The heat flux plot in figure (\ref{fig:hf}), shows the flux of heat through the boundaries of the system. The plot indicates that there is a high heat flux at the corners of the system, which is expected due to the non-uniform diffusion coefficient.

Additionally, the contour plot (Figure \ref{fig:tmd}) of the temperature distribution provides a more detailed visualization of the boundaries and gradients of the system. The contour plot shows that the temperature gradient is highest at the corners of the system, which again is expected due to the non-uniform diffusion coefficient.

Thus, again plots and outputs suggest that the PiNN model is able to accurately predict the temperature distribution and behavior of the system as well. Guaranteed, a further analysis and validation would be necessary to fully assess the performance of the model.

Some limitations of using the PINNs model include the need for a large amount of data to train the model, as well as the sensitivity of the model to the choice of hyperparameters such as the number of layers and neurons in each layer. In addition, the use of the neural network may lead to overfitting, which can result in poor generalization to new data \cite{Hassan2017}. The use of Green's functions in Physics-Informed Neural Networks (PINNs) has several limitations. One of the main limitations is that the analytical expression for the Green's function may not be available for more complex problems. In such cases, numerical or approximate methods may be required to obtain the Green's function, which can be time-consuming and may result in inaccuracies. Another limitation is that the use of Green's functions assumes that the underlying physics of the problem is linear and time-invariant \cite{Skinner}. However, many real-world problems involve non-linear and time-varying physics, which may not be accurately captured by Green's functions.


Furthermore, the PINNs model assumes that the underlying physics is continuous and differentiable, which may not be the case in all scenarios. The implementation of this approach requires knowledge of neural networks, particularly Physics-Informed Neural Networks, and their training. The PiNN model should be trained with the same boundary conditions and initial conditions as the original code. It is essential to select an appropriate architecture for the neural network and perform hyperparameter tuning to optimize the model's performance. Finally, the training of the PINNs model can be computationally expensive, especially for high-dimensional problems, which may limit its practicality in certain applications.