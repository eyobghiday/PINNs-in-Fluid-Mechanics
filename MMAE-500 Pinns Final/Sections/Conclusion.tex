The application of Physics informed neural network provide a powerful tool in determining the solution of the 2D  diffusion equation. In this study a PINN model, which is a type of neural network is shown  partial differential equations (PDEs) with a high degree of accuracy, using both supervised and unsupervised learning. It also discussed the training loss function, which is a measure of how well the model is able to approximate the solution to the PDE. The model involved two different diffusion coefficients and a comparison of the trained and predicted data with test data using contour plots and normal graphs. Despite these challenges, the process of building the model highlights the potential of PINNs to solve complex physical problems, particularly in fields such as fluid dynamics and data science. 

An example code that used the PINNs model to solve a diffusion equation, and generated a heatmap to visualize the solution. The results showed that the PINNs model was able to accurately approximate the solution to the PDE, with a low training loss and high accuracy. Overall, the discussion highlighted the effectiveness of the PINNs model in solving PDEs, and the importance of the training loss function in measuring the accuracy of the model. The example code provided a clear demonstration of the power and versatility of this approach in solving complex problems in physics and engineering. Full access to the codes can be found on github \cite{Eyob2023_Git}, as well as all the documentation accompanying to this project.