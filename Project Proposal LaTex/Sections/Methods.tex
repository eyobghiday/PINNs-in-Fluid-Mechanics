The methodology involves generating a dataset of Navier-Stokes solutions for the flow field around a fluid object using a numerical method, such as finite volume or finite element methods. The dataset should include various geometries and boundary conditions to ensure the model's robustness. Regarding collecting those datasets, it will likely be obtained from the department of the aerospace engineering at Illinois Tech. Worst case scenario we can also do a random data generator method using python or MATLAB. The focus of this project is more on the application side of PINN's and not on how the data it self is obtained. We'll then develop a PINN model to predict the flow field around a fluid object by incorporating the Navier-Stokes equations as a constraint. The PINN model should consist of a neural network that takes as input the fluid object's geometry and boundary conditions and outputs the flow field variables. Training the PINN model using the dataset of Navier-Stokes solutions will be utilised after. During training, the PINN model should satisfy the Navier-Stokes equations as a constraint, and the dataset of Navier-Stokes solutions should be used to compute the loss function. As a result the performance of the PINN model will be evaluated in predicting the flow field around a fluid object using various metrics, such as mean absolute error, root mean square error, and or the correlation coefficient.
