The application of Physics informed neural network provides a powerful tool for determining the solution of the 2D diffusion equation. In this study, a PINN model is presented, which is a type of neural network capable of solving partial differential equations (PDEs) with high accuracy, using both supervised and unsupervised learning. The full 2D diffusion partial differential equation is solved using the model, which is followed by a discussion of the training loss function, a measure of how well the model can approximate the solution to the PDE. The model considered two different diffusion coefficients, and a comparison between the trained and predicted data with test data is carried out using contour plots and normal graphs. Despite the challenges encountered, the process of building the model highlights the potential of PINNs to solve complex physical problems, particularly in fields such as fluid dynamics and data science.

An example code utilizing the PINNs model to solve a diffusion equation generates a heatmap to visualize the solution. The results indicate that the PINNs model can accurately approximate the solution to the PDE, with low training loss and high accuracy. Overall, the discussion emphasizes the effectiveness of the PINNs model in solving PDEs, and the significance of the training loss function in measuring the accuracy of the model. The example code provides a clear demonstration of the power and versatility of this approach in solving complex problems in physics and engineering. For further details, a detailed derivation of the 2D equation can be found in the referenced source \cite{Eyob2023}. Access to the codes and all necessary documentation accompanying this project for future updates is also made accessible through github \cite{Eyob2023_Git}.